% A4, 二段組,フォントサイズ 9bp, 段間 2.36char
\documentclass[fontsize=9bp,twocolumn,column_gap=2.36zw,a4paper,report]{jlreq} 
\usepackage[top=1.20in,bottom=0.95in,left=0.59in,right=0.59in,headheight=0.28in,headsep=0.42in]{geometry}
\usepackage[T1]{fontenc}
\usepackage[utf8]{inputenc}
\usepackage{graphicx}
\usepackage{hyperref}
\usepackage{multirow}
\usepackage{tabularx}
\usepackage{color}
\usepackage{textcomp}
\usepackage{tipa}
\usepackage{amsmath}
\usepackage{amssymb}
\usepackage{amsfonts}
\usepackage{amsxtra}
\usepackage{wasysym}
\usepackage{isomath}
\usepackage{mathtools}
\usepackage{txfonts}
\usepackage{upgreek}
\usepackage{enumerate}
\usepackage{tensor}
\usepackage{pifont}
\usepackage{ulem}
\usepackage{xfrac}
\usepackage{arydshln}
\definecolor{red}{rgb}{1,0,0}

\usepackage{luatexja-fontspec}
% 欧文文字のフォントは「Times New Roman」指定
\setmainfont[Ligatures={Rare,TeX}]{Times-New-Roman}
\setsansfont{Arial}
% 和文文字のフォントは「MS 明朝」指定
\setmainjfont[
    YokoFeatures       = {JFM=jlreq},
    TateFeatures       = {JFM=jlreqv},
    BoldFont           = MS-Mincho,
    BoldFeatures       = {FakeBold=2},
    ItalicFont         = MS-Mincho,
    ItalicFeatures     = {FakeSlant=0.33},
    BoldItalicFont     = MS-Mincho,
    BoldItalicFeatures = {FakeBold=2, FakeSlant=0.33}
]{MS-Mincho}
\setsansjfont[
    YokoFeatures       = {JFM=jlreq},
    TateFeatures       = {JFM=jlreqv},
    BoldFont           = MS-Gothic,
    BoldFeatures       = {FakeBold=2},
    ItalicFont         = MS-Gothic,
    ItalicFeatures     = {FakeSlant=0.33},
    BoldItalicFont     = MS-Gothic,
    BoldItalicFeatures = {FakeBold=2, FakeSlant=0.33}
]{MS-Gothic}
\usepackage{setspace}
\usepackage{caption}
% キャプションのフォントはミディアム
\captionsetup[figure]{font=md}
% キャプションをページ全体での連番にする
\counterwithout{figure}{chapter}
\counterwithout{figure}{section}
% キャプションと参照時の名前の設定
\renewcommand{\figurename}{}
\renewcommand{\thefigure}{図\arabic{figure}}
% キャプションのフォントはミディアム
\captionsetup[table]{font=md}
% キャプションをページ全体での連番にする
\counterwithout{table}{chapter}
\counterwithout{table}{section}
% キャプションと参照時の名前の設定
\renewcommand{\tablename}{}
\renewcommand{\thetable}{表\arabic{table}}
% カラムを自動調整 (X) かつ中央揃え (c)
\newcolumntype{Y}{>{\centering\arraybackslash}X}

\usepackage{unicode-math}
\setmathfont{Cambria-Math}

% ヘッダ 8bp
\newcommand{\header}[1]{\fontsize{8bp}{10bp}\selectfont#1}
% タイトル 16bp 中央揃え
\renewcommand{\title}[1]{\fontsize{16bp}{22bp}\selectfont\centering#1}
% サブタイトルタイトル 14bp 中央揃え
\newcommand{\subtitle}[1]{\fontsize{14bp}{18bp}\selectfont\centering -#1-}
% 筆者 12bp 中央揃え (研究室 名前)を入力
\renewcommand{\author}[2]{\fontsize{12bp}{18bp}\selectfont\centering\vspace{0.3\baselineskip}#1\vspace{0.5\baselineskip}\hspace{1.5em}#2}
% 章 行間はWordデフォルト 改ページしない 12bp 太字
\ModifyHeading{chapter}{
lines=1.15,
allowbreak_if_evenpage=false,
pagebreak=nariyuki,
font=\fontsize{12bp}{16bp}\selectfont\bfseries,
label_format={\thechapter.},
}
% 節 行間はWordデフォルト 改ページしない 10.5bp 太字
\ModifyHeading{section}{
lines=1.15,
allowbreak_if_evenpage=false,
pagebreak=nariyuki,
font=\fontsize{10.5bp}{16bp}\selectfont\bfseries,
label_format={\thesection.},
}

% ヘッダ設定
\usepackage{titleps}
\newpagestyle{header}{
  \sethead
  {\parbox{0.9\textwidth}{\header{
   \ \ 豊田工業高等専門学校 情報工学科\\
   \ \ Department of Information and Computer Engineering,\\
   \ \ National Institute of Technology, Toyota College
   }}}
  {}
  {\parbox{\textwidth}{\header{
   卒業論文\\
   Graduation Thesis\\
   February 22, \the\year % 日にちは都度変更する必要有り
   }}}
   \setfoot
   {}
   {\thepage}
   {}
}

% 参考文献の見出しを「文献」に変更
\renewcommand{\refname}{%
{\fontsize{10.5bp}{16bp}\selectfont\bfseries\centering{文\ \ \ \ \ \ 献}\par}
}
% ヘッダとフッタの bibliography を削除
\renewcommand{\markboth}[2]{}

% 1 ページ目以外はフッタのページ数のみ
\pagestyle{plain}

\begin{document}
% 1 ページ目のみヘッダを挿入
\thispagestyle{header}
% タイトル(サブタイトル)とあらまし,キーワードは二段組にしない
\twocolumn[
{\title コンパイラ学習者のためのYacc用言語サーバの開発\par}
{\author {稲垣研究室} {石部 鳳空}\par}
{\textbf{あらまし}\quad これは豊田高専情報工学科の卒業論文用\LaTeX テンプレートです.提供されるdocxファイルとほぼ同様の見た目になるようにフォントや文字間隔や余白などを調整しました.Lua\LaTeX 処理系の使用を想定しています.\par}
{\textbf{キーワード}\quad \LaTeX, Word,豊田高専,卒業論文,テンプレート
\newline}]

% 和文文字間隔の設定
\ltjsetkanjiskip 0.165em plus 0.165em minus 0.165em
% 欧文和文文字間隔の設定
\ltjsetxkanjiskip 0.33em plus 0.33em minus 0.33em
% 欧文欧文文字感覚の設定
\addfontfeature{LetterSpace=5}

% 本文 (基本的にはタイトルとこれ以降の部分の編集をする)
\chapter{背景}
\section{Yaccとは}

Yaccは高専や大学のコンパイラの講義やコンパイラ開発者,研究者などに広く使用されている一般的なLALR(1)パーサジェネレータである.
Yaccファイルには終端記号と非終端記号を定義し,それらを参照しながらBNFで構文規則と対応する意味動作を記述する.
作成したファイルをYaccで処理すると,内部でオートマトン(実体としては構文解析表である)とスタックを生成し,構文解析関数がこれらを操作して構文解析を実行するため,ユーザはこの関数を呼び出すだけで構文解析が実行できる.\par
コンパイラの授業の演習などでYaccを採用している例として,豊田高専専攻科情報科学専攻の「コンパイラ」\cite{nittc_syllabus}や岡山大学工学部情報系学科「コンパイラ」\cite{okayama_syllabus},電気通信大学情報理工学域「言語処理系論」\cite{uec_syllabus}がある.
またプリンストン大学のコンパイラの講義の教科書である『最新コンパイラ構成技法』\cite{tiger_book}では構文解析器を開発するためにML-Yaccを使用している.

\section{Yaccの周辺ツールの現状}\label{sec:yacc_tools}

Yaccの周辺ツールについて,理解の補助のためのツールとしてLeon Aaron Kaplanによる"yaccviso — a tool for visualizing yacc grammars"\cite{yaccviso}や楠目勝利らによる「コンパイラにおける構文解析過程の視覚化」\cite{parserviso}が提案されている.
これらのツールはYaccファイルに記述した文法を概観したり,Yaccが出力した構文解析器の動作を可視化したりする際に利用される.
また,コーディング時の補助ツールとして,シンタックスハイライトを施すもの\cite{vscode-yacc}や静的解析を行うもの\cite{yash}が存在する.
しかし,これらはVisual Studio Code (VSCode)拡張として提供されているソフトウェアであり,VimやEmacsなどの他の主要なエディタでは同等の支援機能が得られない.
特に静的解析によるコーディング支援を提供するものは\cite{yash}のみであるが,これは現時点(2024年2月8日時点)で73,000回以上インストールされており,Yaccコーディング時の静的解析による支援機能の提供への需要が存在することが分かる.

\begin{figure}[h]
\centering{
\includegraphics[width=0.8\linewidth]{img/lsp.png}
\caption{言語サーバを利用したコーディング支援}\label{fig:lsp}
}
\end{figure}

\chapter{問題と解決法}

セクション\ref{sec:yacc_tools}で述べたように,現時点ではYaccコーディング時の静的解析によるコーディング支援機能の提供を行うソフトウェアはVSCode上でのみ動作し,他の主要なエディタでは同等の支援機能が得られない.\par
そこで本研究では,現在多くの主要なプログラミング言語の実装が存在するLanguage Server Protocol (LSP)\cite{lsp}の仕様に則った言語サーバをYacc用に開発する.
この言語サーバを利用することで,ユーザは使い慣れたエディタでコーディング支援機能を使うことができ,Yaccコーディング時の負担が軽減する.
さらに,新しいエディタが現われた場合でも,エディタ開発者はLSPに則ってクライアントを開発するだけでYaccコーディング時の支援機能をユーザに提供することができる.\par
特定のエディタでのみ記述ができる言語は潜在的なユーザグループを排除しているという指摘\cite{decoupling}がある.
そのためLSPを用いて多様なエディタに対応することで,Yaccユーザやエディタ開発者の負担軽減だけでなく,言語の普及に貢献することもできると考える.
なお,現在Yaccよりも一般に使用されているBison 3.8.1の仕様を元に静的解析器を実装するが,BisonはYaccとの上位互換性のあるソフトウェアのため,便宜上Yaccと表現している.

\begin{figure}[h]
\centering{
\includegraphics[width=0.9\linewidth]{img/lsp-languages-editors.png}
\caption{LSP登場による言語とエディタの関係の変化\cite{lsp_vscode}}\label{fig:lsp-languages-editors}
}
\end{figure}

\chapter{LSPとは}

LSPは2016年にMicrosoftが発表したプロトコルで,エディタと言語サーバ間の通信方法を規定している(\ref{fig:lsp-languages-editors}).
現在は多くの主要なプログラミング言語が公式,非公式を問わず言語サーバを持っており\cite{lsp_impl},プログラミング言語のコーディング環境を整備する際に複数の拡張やプラグインをインストールする必要がないだけでなく,言語クライアントが存在するエディタならばどのエディタでも利用することができることが強みである.\par
近年,GoやTypeScript, Rustといった新しいプログラミング言語が注目を集め利用者が増えているが,エディタやIDEなどの開発者がそれぞれのプログラミング言語用に補完や定義ジャンプ,ホバー時のヒント表示などの多くの支援機能を追加するには多大な労力を要する.
従来は開発ツール毎に支援機能を開発する必要があり,M個の言語に対してN個のエディタが存在し,その実装の数はM\times Nであった.
しかしLSPの登場により,M個の言語サーバとN個のクライアントのみで同様の機能が実現できるようになり,実装の数はM+Nとなった.
このようなことから,LSPは言語のユーザとエディタ開発者双方の負担を軽減することができる.\par
本研究ではLSPのバージョン3.17に準拠する.
\newline

\chapter{Yacc用言語サーバの開発}
\section{開発概要}

本研究ではYacc用言語サーバを開発と複数の言語クライアントの設定を行う.
また言語サーバについて,LSPに則ってリクエストやレスポンスを処理したり,ドキュメントを読み書きしたりするサーバとしての機能だけでなく,静的解析に基づく支援機能をを提供するために静的解析器も実装する.
つまり,開発は主に3つのプログラムからなり,それは言語サーバと言語クライアント,静的解析器である.\par
\ref{tbl:dev}に開発環境を示す.
主要なエディタはすでに言語クライアントが実装済みであり,開発者は設定ファイルへの追記のみで対応できることが多いが,VSCodeは言語クライアントのパッケージを用いて軽微な実装を行う必要があった.
\ref{tbl:dev}中の言語クライアントはVSCodeの言語クライアントである.

\begin{table}[h]
	\caption{開発環境}\label{tbl:dev}
	\centering
	\begin{tabularx}{\linewidth}{|Y||c|c|}
		\hline
		& 言語サーバ & 言語クライアント \\
		\hline\hline
		プログラミング言語 & OCaml 4.14.0 & TypeScript 5.3.3 \\
		\hline
		ビルドシステム & dune 3.7.2 & tsc 5.3.3 \\
		\hline
		ランタイム & OCaml 4.14.0 & Node.js v21.6.1 \\
		\hline
		OS & \multicolumn{2}{c|}{Arch Linux x86\_64} \\
		\hline
		CPU & \multicolumn{2}{c|}{Intel i7-10710U} \\
		\hline
	\end{tabularx}
\end{table}

\section{言語クライアントの実装・設定}

VSCodeで動作する言語クライアントをVSCode拡張として実装した.そこではMicrosoftが開発しているvscode-languageclientというnpmパッケージを使用し,VSCodeの拡張機能としての設定や言語サーバとの接続方法などの設定を記述した.
また,VimとEmacsでも動作されるために,Yaccファイルを開いた状態で言語サーバと通信を行うように設定ファイルに記述を追加した.

\section{言語サーバの通信}

LSPではJSON-RPCを用いて通信をするが,通信方式は指定されていない.
しかしLSPにおいては,同一のコンピュータでクライアントとサーバが動作することが通常であるため,今回は同一コンピュータ上で標準入出力による通信を確立することとした.
実際にVSCodeの言語クライアントのパッケージでは標準入出力だけでなく,パイプ通信やソケット通信,IPC通信が選択できるが,VimやEmacsの言語クライアントのプラグインの多くが標準入出力での通信のみをサポートしている.\par
LSPの仕様に則って,ドキュメント(ファイル)はクライアントではなくサーバ側のハッシュ表で管理することとした.
ドキュメントが開かれると,クライアントはサーバにドキュメントの内容を含むドキュメント情報を送信する.
このリクエストを処理する際に,サーバがドキュメント情報をハッシュ表に格納する.
これ以降のドキュメントの編集では,差分のみがサーバに送信されるため,差分をハッシュ表に格納されている情報に適用することで,サーバはクライアントが開いているドキュメントと同期するように実装した.

\begin{figure}[h]
\centering{
\includegraphics[width=1\linewidth]{img/language_server_lifecycle.png}
\caption{言語サーバのライフサイクル}\label{fig:language_server_lifecycle}
}
\end{figure}

\section{言語サーバのライフサイクル}

言語サーバのライフサイクルを\ref{fig:language_server_lifecycle}に示す.
初期化と起動という2つの状態を保持し,これらの状態に応じてリクエストを受理してレスポンスを返却するか,状態を遷移させてエラーレスポンスを返却するかなどを判断する.\par
LSPの通信でやりとりされるメッセージには,request, response, notificationの3種類がある.
requestはresponseを必要とし,notificationはresponseを必要としないものである.
クライアントとサーバのどちらからも,どのメッセージでも送受信できる.
クライアントは支援機能に関する通信を行う前に,initialize requestを送信する.
これに対してサーバはresponseを返し,クライアントはinitialized notificationを送信する.
この一連の通信でクライアントとサーバはそれぞれのcapabilitiesを送り合う.
これによって,双方はそれぞれが対応している支援機能などについて知ることができる.
これが完了すると,サーバはクライアントから送信された補完や定義ジャンプなどの支援機能に関わるrequestを受理するようになる.

\section{Yaccの静的解析器}

Bison実装の字句解析器と構文解析器の部分をocamllex, Menhirを用いてOCamlへの移植を行った.
これによってBisonの仕様に則った字句解析器と構文解析器を得られた.
また,構文解析の結果の出力として抽象構文木を得るように実装をすることで,静的解析が可能となった.\par
完全な入力を想定している通常のコンパイラと異なり,言語サーバは多くの場面で不完全なソースコードが入力として渡される.
抽象構文木を走査して型検査や変数の定義と使用の検査などを行うため,どのような入力であっても構文木を得ることができなければ十分なコーディング支援を提供することが困難である.
そのため,不完全なエラー回復をして抽象構文木を得る必要がある.
エラー回復(修復)戦略としては,特殊なerror記号を使用した局所的なエラー回復や,Burke-Fisherエラー修復などが考えられるが,前者は対応できる状況に限界があり,後者は2つのスタックを使ってトークンを操作しなければいけないため実装が複雑になるという問題がある.
そこで本研究ではMenhirのInspection APIを利用することとした.
Inspection API はLR構文解析器のオートマトンやスタックの状態にアクセスすることができるため,比較的汎用なエラー回復が実現できる.\par
静的解析の流れを\ref{fig:static_analysis}に示す.
今回,実装した部分を青色の文字で表している.

\begin{figure}[h]
\centering{
\includegraphics[width=1\linewidth]{img/static_analysis.png}
\caption{静的解析の流れ}\label{fig:static_analysis}
}
\end{figure}

\chapter{実行例}

コード診断をVSCode, Vim, Emacsで行っている様子をそれぞれ\ref{fig:vscode},\ref{fig:vim},\ref{fig:emacs}に示す.
ここでは「\%」という入力が不正であることと,予期しない位置EOFがあることを構文解析の過程で検出し,LSPのtextDocument/didSaveメソッドによってエディタに報告している.

\begin{figure}[h]
\centering{
\includegraphics[width=0.9\linewidth]{img/vscode.png}
\caption{VSCodeでのコード診断実行例}\label{fig:vscode}
}
\end{figure}
\begin{figure}[h]
\centering{
\includegraphics[width=0.9\linewidth]{img/vim.png}
\caption{Vimでのコード診断実行例}\label{fig:vim}
}
\end{figure}
\begin{figure}[h]
\centering{
\includegraphics[width=0.9\linewidth]{img/emacs.png}
\caption{Emacsでのコード診断実行例}\label{fig:emacs}
}
\end{figure}

\chapter{考察}

VSCodeとVim, Emacsの言語クライアントを使用し,今回作成した言語サーバを実行するように設定したことで,異なるエディタで同様の支援機能が容易に得られることが確認できた.
また,Bisonの仕様や実装を元に字句解析器と構文解析器を作成し,構文解析器の出力として抽象構文木を構築することで,静的解析を可能にした.
今回は不完全な入力に対して可能な限りエラー回復を試みて抽象構文木を構築し,その過程で見つかった構文エラーなどをLSPのtextDocument/diagnosticメソッドによってクライアントに送信している.
\newline

\chapter{今後の課題}

今回記述したパーサジェネレータのファイルはBisonのものの移植のため,error記号によるエラー回復を前提としており,文法に対応する意味動作に副作用がある場合が多いが,エラー回復のパターンを増やしたり,より安全にエラー回復を行ったりするためには,意味動作を副作用の無いものに変更する必要があると推察する.
また,意味解析器を実装し,コード補完や定義ジャンプなどサポートする機能を増やして利便性を高めていく.
その後,支援機能に関する計算を並列化するなどして効率化を図りたい.\par
このように問題を解決したり機能を追加したりすることで,言語サーバを十分実用に耐え得るものにしていき,OSSとして公開しユーザの獲得や継続的な開発を実現したい.
\newline

\chapter{謝辞}

本研究の遂行にあたり,指導教員として多大なご指導を賜りました稲垣宏教授(豊田工業高等専門学校 情報工学科),ご多忙のなか相談に応じてくださりご助言を頂いた内山慎太郎氏(豊橋技術科学大学 大学院 工学研究科 情報・知能工学専攻 博士後期課程 応用数理ネットワーク研究室)に感謝申し上げます.

\begin{thebibliography}{99}
\begin{spacing}{0}
	\bibitem{nittc_syllabus}{高専Webシラバス, “コンパイラ”, \url{https://syllabus.kosen-k.go.jp/Pages/PublicSyllabus?school_id=23&department_id=25&subject_code=95018&year=2017&lang=ja}, 2023年10月18日閲覧}
	\bibitem{okayama_syllabus}{Nobuya WATANABE, “コンパイラ(Compilers)”, \url{http://www.arc.cs.okayama-u.ac.jp/~nobuya/lecture/compiler/}, 2023年10月18日閲覧}
	\bibitem{uec_syllabus}{電気通信大学 シラバスWeb公開システム, “シラバス参照”, \url{http://kyoumu.office.uec.ac.jp/syllabus/2023/31/31_21124118.html}, 2023年10月18日閲覧}
	\bibitem{tiger_book}{Andrew W. Appel,Modern compiler implementation in ML,Cambridge University Press,New York, Cambridge,2008.(アンドリュー・W・エイペル 神林靖・滝本宗宏(訳), 最新コンパイラ構成技法, 翔泳社, 東京, 2020.)}
	\bibitem{yaccviso}{Leon Aaron Kaplan, “yaccviso — a tool for visualizing yacc grammars”, 2006.}
	\bibitem{parserviso}{楠目 勝利, 佐々 政孝. “コンパイラにおける構文解析過程の視覚化”, 全国大会講演論文集, 第55回, ソフトウェア科学・工学, pp448-449, 1997.}
	\bibitem{vscode-yacc}{Visual Studio Marketplace, “VSCode-YACC”, \url{https://marketplace.visualstudio.com/items?itemName=carlubian.yacc}, 2023年10月19日閲覧}
	\bibitem{yash}{Visual Studio Marketplace, “Yash”, \url{https://marketplace.visualstudio.com/items?itemName=daohong-emilio.yash}, 2023年10月6日閲覧}
	\bibitem{lsp}{Official page for Language Server Protocol, “Language Server Protocol Specification - 3.17”, \url{https://microsoft.github.io/language-server-protocol/specifications/lsp/3.17/specification/}, 2023年10月19日閲覧}
	\bibitem{lsp_impl}{Official page for Language Server Protocol, “Implemantations”, \url{https://microsoft.github.io/language-server-protocol/implementors/servers/}, 2023年10月19日閲覧}
	\bibitem{lsp_vscode}{Language Server Extension Guide | Visual Studio Code Extension API, \url{https://code.visualstudio.com/api/language-extensions/language-server-extension-guide}, 2024年2月8日閲覧}
	\bibitem{decoupling}{Bünder, H. “Decoupling Language and Editor - The Impact of the Language Server Protocol on Textual Domain-Specific Languages.” International Conference on Model-Driven Engineering and Software Development, no.10.5220/0007556301290140, pp.129-140, Prague, Czech Republic, 2019.}
	\bibitem{vscode-languageclient}{npm, “vscode-languageclient”, \url{https://www.npmjs.com/package/vscode-languageclient}, 2023年10月19日閲覧}
	\bibitem{menhir}{Menhir Reference Manual (version 20231231), “Inspection API”, \url{https://gallium.inria.fr/~fpottier/menhir/manual.html#sec64}, 2024年1月28日閲覧}
\end{spacing}
\end{thebibliography}

\end{document}
